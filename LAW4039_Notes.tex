\documentclass{article}
\usepackage{adi}
\usepackage{amsmath}

\title{LAW4039 Notes}
\author{Adithya C. Ganesh}

\begin{document}

\maketitle

\tableofcontents

\newpage

\section{Lecture 1: Society, Institutions, Risk, and AI}

\subsection{R1: Why the US Went to War in Vietnam}

https://www.fpri.org/article/2017/04/united-states-went-war-vietnam/

\subsection{R2: Five years after Deepwater Horizon, what has the disaster taught us?}

https://www.minnpost.com/earth-journal/2015/04/five-years-after-deepwater-horizon-what-has-disaster-taught-us/

\subsection{R3: A Declaration of the Independence of Cyberspace (1996)}

\subsection{R4: The Great AI Awakening, N.Y. Times Magazine (Dec. 14, 2016)}

\subsection{R5: Regulating AI Systems: Risks, Challenges, Competencies, and Strategies}

\subsection{R6: Russell and Norvig, AI (2015)}

\subsection{R7: Suleyman and Musk: Humans Must Become Cyborgs to Avoid AI Domination}

\subsection{R8: Chui: Applying AI for Social Good, Discussion Paper, McKinsey and Co}

\subsection{Lecture Notes}

The challenge for technologists: learning a little bit of law during these 10 weeks.

\subsubsection{A tragic instrusion}

The hardest cases to discuss and analyze: the death penalty cases (these ones go to California Supreme Court).  Example: intrudred enters the home of a woman who's home with her four year old son.  Mother was stabbed, and bleeds to death.

Important questions:

\begin{itemize}
  \item How to structure a six-picture ``show-up'' for the child at the push of a button?
  \item Do we use pervasive surveillance capacity into an ML-enabled infrastructure for security?
\end{itemize}

Death penalty cases are not-unlike analyzing the ethics of AI systems, where one must weigh the balance between different foundational ethical principles.

Two AI based approaches:
\begin{itemize}
  \item GANs to generate images based on textual descriptions.
  \item Convolutional neural networks for facial recognition / person tracking. 
\end{itemize}

But there are concerns:
\begin{itemize}
  \item How accurate is the technology?
  \item What is the line between police and non-police?
\end{itemize}

\subsection{Why are we here?  Because AI is not just theory\dots}

AlphaZero works across games.  5-7 years ago, most of these feats were highly domain specific.

Kasparov: ``I can't disguise my satisfaction that it plays with a very dynamic style\dots much like my own.''

\subsection{What is AI?}

\begin{itemize}
  \item Capacity to undertake functions that, if performed by a human, would generally be understood to require intelligence`` (Russell and Norvig)
  \item Definition of ''intelligence``  ''Ability to learn or understand or to deal with new or trying situations``
\end{itemize}

It's worth differentiating between a few types of AI: {\it domain-specific AI}, which is supervised ML attached to some application; {\it simulated AGI}, (e.g. Echo / Google Home, etc.).

\subsection{What is regulation?}

For instance: how fast can you go on 280?  While the Autobahn has no speed limits, 70-90 is a reasonable speed when 

\begin{itemize}
  \item We define the term broadly: How society defines legitimate authority to structure relations among people, organizations, information, and the physical world.
  \item Not just specific rules, imposed by agencies, governing the design, operation, or use of technology.
\end{itemize}

Governance intersects with politics, socioeconomic behavior, specifically including:
\begin{itemize}
  \item Constitutions
  \item Statutes
  \item Judge-made law (e.g. the common law)
  \item Agency decisions
  \item Norms
\end{itemize}

\subsection{A (straightforward) problem AI can help address}

\begin{itemize}
  \item 210 languages spoken in California courts.
  \item Interpreters provide services in courtrooms, but not enough multi-lingual staff over the counter.
  \item 2018: We began a pilot project to provide translation over-the-counter to limited-English court users, using tablets
  \item AI could help enhance: access to justice (but als omedical care, education).
\end{itemize}

\subsection{Deepwater Horizon: disaster}

\begin{itemize}
  \item Offshore drilling: Hole, metal casing, cement ``slugs,'' drilling fluid pumped down to balance pressure; drilling operators cement well, then replaced by rig to extract oil.
  \item But problems lurked: Well had a fracture, troubling readings ignored, too much mud displaced, blowout preventer failed.
  \item 9:49pm, April 20, 2010: First explosisons occurred.
  \item Cuellar at the WH, helping to reconcile work of NOAA and FDA to assess and mitigate impact of the disaster on seafood.
\end{itemize}

\end{document}
