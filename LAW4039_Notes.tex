\documentclass{article}
\usepackage{adi}
\usepackage{amsmath}
\usepackage{todonotes}

\title{LAW4039 Notes}
\author{Adithya C. Ganesh}

\begin{document}

\maketitle

\tableofcontents

\newpage

\section{Lecture 1: Society, Institutions, Risk, and AI}

\subsection{R1: Why the US Went to War in Vietnam}

https://www.fpri.org/article/2017/04/united-states-went-war-vietnam/

\subsection{R2: Five years after Deepwater Horizon, what has the disaster taught us?}

https://www.minnpost.com/earth-journal/2015/04/five-years-after-deepwater-horizon-what-has-disaster-taught-us/

\subsection{R3: A Declaration of the Independence of Cyberspace (1996)}

\subsection{R4: The Great AI Awakening, N.Y. Times Magazine (Dec. 14, 2016)}

\subsection{R5: Regulating AI Systems: Risks, Challenges, Competencies, and Strategies}

\subsection{R6: Russell and Norvig, AI (2015)}

\subsection{R7: Suleyman and Musk: Humans Must Become Cyborgs to Avoid AI Domination}

\subsection{R8: Chui: Applying AI for Social Good, Discussion Paper, McKinsey and Co}

\subsection{Lecture Notes}

The challenge for technologists: learning a little bit of law during these 10 weeks.

\subsubsection{A tragic instrusion}

The hardest cases to discuss and analyze: the death penalty cases (these ones go to California Supreme Court).  Example: intrudred enters the home of a woman who's home with her four year old son.  Mother was stabbed, and bleeds to death.

Important questions:

\begin{itemize}
  \item How to structure a six-picture ``show-up'' for the child at the push of a button?
  \item Do we use pervasive surveillance capacity into an ML-enabled infrastructure for security?
\end{itemize}

Death penalty cases are not-unlike analyzing the ethics of AI systems, where one must weigh the balance between different foundational ethical principles.

Two AI based approaches:
\begin{itemize}
  \item GANs to generate images based on textual descriptions.
  \item Convolutional neural networks for facial recognition / person tracking. 
\end{itemize}

But there are concerns:
\begin{itemize}
  \item How accurate is the technology?
  \item What is the line between police and non-police?
\end{itemize}

\subsection{Why are we here?  Because AI is not just theory\dots}

AlphaZero works across games.  5-7 years ago, most of these feats were highly domain specific.

Kasparov: ``I can't disguise my satisfaction that it plays with a very dynamic style\dots much like my own.''

\subsection{What is AI?}

\begin{itemize}
  \item Capacity to undertake functions that, if performed by a human, would generally be understood to require intelligence`` (Russell and Norvig)
  \item Definition of ''intelligence``  ''Ability to learn or understand or to deal with new or trying situations``
\end{itemize}

It's worth differentiating between a few types of AI: {\it domain-specific AI}, which is supervised ML attached to some application; {\it simulated AGI}, (e.g. Echo / Google Home, etc.).

\subsection{What is regulation?}

For instance: how fast can you go on 280?  While the Autobahn has no speed limits, 70-90 is a reasonable speed when 

\begin{itemize}
  \item We define the term broadly: How society defines legitimate authority to structure relations among people, organizations, information, and the physical world.
  \item Not just specific rules, imposed by agencies, governing the design, operation, or use of technology.
\end{itemize}

Governance intersects with politics, socioeconomic behavior, specifically including:
\begin{itemize}
  \item Constitutions
  \item Statutes
  \item Judge-made law (e.g. the common law)
  \item Agency decisions
  \item Norms
\end{itemize}

\subsection{A (straightforward) problem AI can help address}

\begin{itemize}
  \item 210 languages spoken in California courts.
  \item Interpreters provide services in courtrooms, but not enough multi-lingual staff over the counter.
  \item 2018: We began a pilot project to provide translation over-the-counter to limited-English court users, using tablets
  \item AI could help enhance: access to justice (but als omedical care, education).
\end{itemize}

\subsection{Deepwater Horizon: disaster}

\begin{itemize}
  \item Offshore drilling: Hole, metal casing, cement ``slugs,'' drilling fluid pumped down to balance pressure; drilling operators cement well, then replaced by rig to extract oil.
  \item But problems lurked: Well had a fracture, troubling readings ignored, too much mud displaced, blowout preventer failed.
  \item 9:49pm, April 20, 2010: First explosisons occurred.
  \item Cuellar at the WH, helping to reconcile work of NOAA and FDA to assess and mitigate impact of the disaster on seafood.
\end{itemize}

Importantly, note that:
\begin{itemize}
  \item Blowout preventer failed; no clear solution other than to drill a replacement well, which would take months
  \item Hard to understand who was responsible.
  \item 5M gallons of oil released over 87 days.
  \item 11 lost lives and many more sickened
  \item Oil spill affected water quality, fisheries, etc.
  \item Financial impact as of 2018: circa \$65B (nearly \$9B impact on fisheries alone).
\end{itemize}

\subsection{Deepwater Horizon: Aftermath}
\begin{itemize}
  \item ``oil and gas industry has not retreated to safety.  Instead, it has expanded its technological horizon in ways that make it harder to foresee the complex interactions between drilling technologies, inevitable human errors, and the ultra-deepwater environment''
  \item More layers of abstraction?  Seems like sketchy incentives.
\end{itemize}

\subsection{Deepwater Horizon: Why does this matter?}
\begin{itemize}
  \item Could some kind of AI system have helped prevent the disaster?
  \item Maybe: accidents and the result harms can be reasons to innovate.
  \item Can we at least expect govt. to handle thoughtfully the biggest risks they perceive?
\end{itemize}

\subsection{Second case study: Vietnam War - Puzzles}
\begin{itemize}
  \item Why did the U.S. intervene on the other side of the world? (Where the French had failed)
  \item Why did the U.S. persist in a course of action so likely to result in failure?
  \item How could Vietnam win against a technologically and economically superior adversary?
  \item How does this episode tell us about how governments make decision (even when they have so much at stake) and about geopolitics?
\end{itemize}

\subsection{Vietnam: French influence and internecine struggles}
\begin{itemize}
  \item French influence: By 1860s, French troops engage in battle for control of ``Indochina,'' are awarded 3 provinces by Emperor Tu Dug as a concession
  \item French power consolidated: By 1879, French (having taken over Cambodia and Vietnam) name first colnial governon and claim ``Annam and Tonkin'' as protectorate.
\end{itemize}

\subsection{U.S. interlude: Roosevelt and anti-colonialism}
\begin{itemize}
  \item U.S. strategic interest not served by advancing colonialism
  \item Franklin Roosevelt had personal concerns
\end{itemize}

\todo{Finish this}

\subsection{Vietnam War: A tenuous post-colonial compromise (1954)}
\begin{itemize}
  \item Peace conference in Geneva resulted in the agreement that the French positions would be evacuated completely
\end{itemize}

\subsection{The Geneva Accords break down}
\begin{itemize}
  \item Accords quickly unravel:
    \begin{itemize}
      \item French encourage independence of South Vietnam, using Bao Dai.
      \item Bao Dai (with advice of the French) appoints Diem premier, then President.  Rejects Geneva Accords.
      \item U.S. did not sign and did not feel bound to them; intense domestic concern over the victory of Mao and allies in China.
    \end{itemize}
\end{itemize}

\subsection{Kennedy \& Ho: Evolving priorities}
\begin{itemize}
  \item Kennedy - concern with 1964 election.
  \item Ho - consolidation, Internal political divisions.
\end{itemize}

\subsection{From Kennedy to Johnson: Change \& continuity}
\begin{itemize}
  \item Continuity: powerful faction concerned about U.S. credibility
  \item Change: different political calculations (Johnson v. Kennedy): ``Perhaps the most complicated figure in 20th century history.''
\end{itemize}

\subsection{Johnson's political \& org constraints}
\begin{itemize}
  \item Expansive domestic program
  \item Looming presidential election
  \item Key advisers preferred expanding the war
  \item Political opponents sought to target Johnson Administration
\end{itemize}

\subsection{Self-reinforcing strategic traps}
\begin{itemize}
  \item North Vietnamese and VC strategy: Inflict losses on the U.S. sufficient to persuade it to leave or negotiate.
  \item Washington D.C. strategy: pursue a course minimizing risks that U.S. will lose credibility; leaving in the wake of attacks on troops would all but guarantee such a loss in credibility
\end{itemize}

\subsection{Johnson's decision}
\begin{itemize}
  \item Johnson's domestic program
  \item Why does Johnson accept?
\end{itemize}

\subsection{U.S. context: Domestic impact}
\begin{itemize}
  \item Draft
  \item Economic impact
  \item Mounting casualties
  \item Growing social unrest
\end{itemize}

\subsection{By 1966: McNamara relents}
McNamara: a complex figure, who did math.  The 8th U.S. Secretary of Defense.  He would be able to understand the mathematical basis of AI that we're talking about here.
\begin{itemize}
  \item Former architect of war begins to see no change.
  \item McNamara writes key memo, but keeps it secret.
    \begin{itemize}
      \item ``The picture of the world's greatest superpower killing or seriously injuring 1000 noncombatants a week, while trying to \dots is not a pretty one.
    \end{itemize}
  \item Out of loyalty to LBJ, continues to defend war.
  \item Personal dimension.
\end{itemize}

\subsection{Mid-1970s: Saigon Falls}
\begin{itemize}
  \item Saigon's deterioriating position.
  \item Chaos as Saigon faills in 1975
\end{itemize}

\subsection{Why the delay in stopping? Countries are a ``they,'' not an it}

\subsection{Vietnam War: Implications}
\begin{itemize}
  \item Severe economic costs and political isolation for Vietnam.
  \item Collapse of South Vietnam.
  \item Political turmoil and devastation in Southeast Asia.  Laos is perhaps the most heavily bombed country in history.
  \item Deaths: up to 2 million Vietnamese civilians, 1.1 million N. Vietnamese military, 250K S. Vietnamese military, 58K U.S. military.
\end{itemize}

Operation Igloo White: an early military operation of AI (\$1B).

\subsection{Responding to risk, conflict, and innovation: Preliminary implications}
\begin{itemize}
  \item Deepwater Horizon
    \begin{itemize}
      \item Harm from neglect of systemic safety issues
      \item Overlapping responsibility and lack of clarity in accountability, despite laws
      \item Deep contradictions in how society benefits from and is harmed by technology
      \item Political economy lurks in the background
    \end{itemize}
  \item Vietnam war: interaction of history and geopolitical circumstances, technology, domestic political pressures, and organizational realities = an outcome no one person or organization wanted.
\end{itemize}

\subsection{Preliminary implications of case studies (cont\dots)}

\begin{itemize}
  \item Early internet
  \item Explosive growth
\end{itemize}

\subsection{Simple summary of case study implications}
\begin{itemize}
  \item Small choices have big consequences
  \item Geopolitics lurk in the background at nearly every turn
  \item ``Emergence'' is a crucial factor
  \item Virtually all big societal dilemmas have both a legal and an organizational dimension (e.g. Vietnam).
\end{itemize}

Themes make the case for ``regulatory thinking''\dots

\subsection{Defining AI: ``The behavior of human beings''}

There are two broad kinds of AI problems: {\it recognition}, vs. {\it cognition}.  The former is less ``explainable'' and the latter is more ``explainable.''  Slowing down deliberately can enable ``system 2'' modes of thinking.

\todo{Alpha Zero is definitely both}

\subsection{Execute a class w/ these goals: Four legs of a stool}

\begin{itemize}
  \item Lectures
  \item Speakers
  \item Group projects
  \item Your own work
  \item Attendance
\end{itemize}

\subsection{Preview: substantive topics}
\begin{itemize}
  \item Key people: Herbert Simon (AI as decision science), Frank Fukuyama (trust and power are central to politics and the state), Ed Feigenbaum (Cognition, recognition; not just machine learning)
  \item Intellectual influences: Judith Shklar, Charles Perrow, Arlie Hochschild
\end{itemize}



\end{document}
