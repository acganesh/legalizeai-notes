\documentclass{article}
\usepackage{adi}
\usepackage{amsmath}

\title{LAW4039 Notes}
\author{Adithya C. Ganesh}

\begin{document}

\maketitle

\tableofcontents

\section{Lecture 1: Society, Institutions, Risk, and AI}

\subsection{R1: Why the US Went to War in Vietnam}

https://www.fpri.org/article/2017/04/united-states-went-war-vietnam/

\subsection{R2: Five years after Deepwater Horizon, what has the disaster taught us?}

https://www.minnpost.com/earth-journal/2015/04/five-years-after-deepwater-horizon-what-has-disaster-taught-us/

\subsection{R3: A Declaration of the Independence of Cyberspace (1996)}

\subsection{R4: The Great AI Awakening, N.Y. Times Magazine (Dec. 14, 2016)}

\subsection{R5: Regulating AI Systems: Risks, Challenges, Competencies, and Strategies}

\subsection{R6: Russell and Norvig, AI (2015)}

\subsection{R7: Suleyman and Musk: Humans Must Become Cyborgs to Avoid AI Domination}

\subsection{R8: Chui: Applying AI for Social Good, Discussion Paper, McKinsey and Co}

\subsection{Lecture Notes}

The challenge for technologists: learning a little bit of law during these 10 weeks.

\subsubsection{A tragic instrusion}

The hardest cases to discuss and analyze: the death penalty cases (these ones go to California Supreme Court).  Example: intrudred enters the home of a woman who's home with her four year old son.  Mother was stabbed, and bleeds to death.

Important questions:

\begin{itemize}
  \item How to structure a six-picture ``show-up'' for the child at the push of a button?
  \item Do we use pervasive surveillance capacity into an ML-enabled infrastructure for security?
\end{itemize}

Death penalty cases are not-unlike analyzing the ethics of AI systems, where one must weigh the balance between different foundational ethical principles.

Two AI based approaches:
\begin{itemize}
  \item GANs to generate images based on textual descriptions.
  \item Convolutional neural networks for facial recognition / person tracking. 
\end{itemize}


\end{document}
